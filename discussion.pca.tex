% ---------------------------------------------------------
% Project: PhD KAPPA
% File: discussion.pca.tex
% Author: Andrea Discacciati
%
% Purpose: BMI and prostate cancer (discussion)
% ---------------------------------------------------------

\section{Body mass index and prostate cancer incidence and mortality}

\subsection{Major findings and comparison with literature results}

\subsubsection{BMI during middle-late adulthood}

In \citetalias{discacciati_body_2011} and \citetalias{discacciati_body_2012} we observed a dual association between BMI during middle-late adulthood an prostate cancer: an increased risk of advanced and fatal prostate cancer, and a decreased risk of localized prostate cancer (comparing overweight and obese men to normal-weight men). These findings support the hypothesis of etiological heterogeneity of prostate cancer in relation to obesity \citep{discacciati_lifestyle_2014}.

In \citetalias{discacciati_body_2011} we observed an inverse--U shaped relation between BMI and incidence of localized prostate cancer. This result was also observed in the updated analyses with a longer follow-up period and, noteworthy, it was robust to different modeling strategies of the exposure. Although no evidence of non-linearity was observed in \citetalias{discacciati_body_2012}, the updated meta-analysis showed a similar dose--response relation to that reported in \citetalias{discacciati_body_2011}. Furthermore, the effect size estimates were remarkably similar. A lower localized prostate cancer incidence was observed in particular for overweight and obese men as compared with normal-weight men. The decreased incidence for underweight men was much less pronounced.

An increased incidence of advanced prostate cancer was observed both in \citetalias{discacciati_body_2011} and in \citetalias{discacciati_body_2012}. Results were again consistent, especially considering the updated analyses carried out in this thesis, where a 7--11\% increased IR was observed for every 5-unit increment in BMI. 

Consistently with the results for advanced prostate cancer, a 12\% increased MR was observed for death due to prostate cancer in \citetalias{discacciati_body_2011}. The increased number of cases due to the extended follow-up time contributed in making the estimated MRR slightly more precise. In addition, the worse survival associated with increased BMI levels was also observed when analyzing the percentiles of age at prostate cancer death using quantile regression for censored data.

These results are in line with recent epidemiologic literature. Higher BMI levels were observed to be associated with lower localized prostate cancer incidence in a large dose--response meta-analysis, where a non-linear relation was observed \citep{wcrf_continuous_2014}.  Interestingly, an inverse--U shaped association between BMI and the incidence of total prostate cancer was also found in a very large population-based cohort study of 5.24 million adults in the United Kingdom \citep{bhaskaran_bodymass_2014}. Although the analyses were not carried out by specific subtypes of prostate cancer, one might speculate that the presumably high proportion of localized cases conferred to the association an inverse--U shape.

BMI was positively linearly associated with advanced prostate cancer in many, but not all, of the cohort studies that we included in the meta-analysis, as shown in figure \ref{fig:ss_adv}. However, the results from a recent dose--response meta-analysis of 24 cohort studies are in agreement with our findings \citep{wcrf_continuous_2014}. Comparable results were also observed in a ``high versus low'' meta-analysis \citep{zhang_impact_2015}. 

Lastly, a positive association between BMI and prostate cancer mortality, similar in magnitude to what we observed in the COSM, was reported in a dose--response meta-analysis of 6 population-based cohort studies [RR for every 5-unit increment in BMI: 1.15 (95\% CI: 1.06--1.25)]\citep{cao_body_2011}. A large study, not included in the aforementioned meta-analysis, also observed a positive association between BMI and prostate cancer MR \citep{haggstrom_prospective_2012}.  Consistently with these findings, BMI was observed to be associated with a higher IR of biochemical recurrence \citep{bassett_impact_2005, strom_obesity_2005, strom_influence_2006, efstathiou_influence_2007, palma_obesity_2007, ly_association_2010, cao_body_2011} and with a higher rate of prostate cancer--specific mortality in many \citep{efstathiou_obesity_2007, gong_obesity_2007, ma_prediagnostic_2008, cao_body_2011, cantarutti_body_2015}, but not all studies \citep{bonn_body_2014}.  

%TODO: gene effect modification?

\subsubsection{BMI during early adulthood}

In \citetalias{discacciati_body_2011}, no evidence of an association  between BMI during early adulthood and incidence of localized prostate cancer was observed. This lack of evidence remained also after extending the follow-up until 31 December 2011. Although we observed inverse associations between BMI at age 30 years and the risk of advanced and fatal prostate cancer, these results were very weak.

Comparison of these results with other studies is complicated due to the heterogeneity in the time window considered for the exposure \citep{robinson_systematic_2008, sutcliffe_prostate_2013}. A systematic review and dose--response meta-analysis of 16 studies on BMI measured at 18--29 years of age and total prostate cancer observed a 6\% increased risk for every 5-unit increment in early-adult BMI [RR: 1.06 (95\% CI: 0.99--1.14)]  \citep{robinson_systematic_2008}. In particular, the pooled RR for the 5 studies reporting BMI during the ages 25--29 years was equal to 1.14 (95\% CI: 1.00--1.30), in contrast with our findings. Studies analyzing BMI `during college' age and `during the twenties' reported weaker associations.

Only a few studies investigated the association between BMI during early adulthood by subtype of the disease. For localized prostate cancer, findings were inconsistent \citep{schuurman_anthropometry_2000, littman_anthropometrics_2007, wright_prospective_2007}. 
Similarly, results for advanced prostate cancer were heterogeneous, ranging from inverse \citep{moller_body_2015} to direct associations \citep{schuurman_anthropometry_2000, dalmaso_prostate_2004}. However, the majority of the studies reported very little evidence of an association \citep{giles_early_2003, robinson_obesity_2005, littman_anthropometrics_2007, wright_prospective_2007, moller_lifetime_2013}. Not surprisingly, these inconsistent results were also observed for prostate cancer mortality \citep{wright_prospective_2007, burton_young_2010, moller_lifetime_2013, moller_body_2015}.


\subsection{Biological mechanisms}

\subsubsection{BMI during middle-late adulthood}

Possible biological mechanisms that could explain the associations between BMI and prostate cancer incidence and mortality observed in \citetalias{discacciati_body_2011} and \citetalias{discacciati_body_2012} are still partially unclear. However, three pathways in particular have been proposed in the literature: the insulin/insulin-like growth factor 1 (IGF-1), the sex hormones, and the inflammation pathway \citep{wiren_androgens_2008, roberts_biological_2010, renehan_adiposity_2015}. 

First, obesity is associated with hyperinsulinemia which in turn, due to decreased levels of IGF-binding proteins 1 and 2, increases the circulating amounts of bioactive IGF-1 \citep{nam_effect_1997}, a growth factor with a pathogenic role in many cancers \citep{roberts_biological_2010}. High levels of IGF-1, in particular, have been associated with increased prostate cancer incidence in 2 meta-analyses \citep{renehan_insulinlike_2004, rowlands_circulating_2009} and in a large pooling project of 12 prospective studies \citep{roddam_insulinlike_2008}. However, a shortcoming of this explanation is that mean IGF-1 levels have been observed to be non-linearly associated with BMI, peaking at 24--26 \kgmsq{} \citep{yamamoto_relationship_1993, lukanova_nonlinear_2002}. Interestingly, this was roughly the BMI range where the highest incidence of localized prostate cancer was observed in \citetalias{discacciati_body_2011} (figure \ref{fig:paper1_1997}) and in the updated dose--response meta-analysis (figure \ref{fig:pool_loc}).

Second, obesity is linked with decreased androgen levels \citep{lima_decreased_2000}. No evidence of an association between androgens and incidence of total prostate cancer was observed in a large collaborative analysis of 18 prospective studies \citep{endogenoushormonesandprostatecancercollaborativegroup_endogenous_2008}. At the same time, in line with our findings, lower concentrations of free testosterone were observed to be linked with a decreased incidence of non-aggressive well-differentiated prostate cancer and with an increased incidence of aggressive low-differentiated prostate cancer in two prospective cohort studies \citep{platz_sex_2005, severi_circulating_2006}. Furthermore, it has been observed that, among men diagnosed with prostate cancer, those with lower testosterone levels have a higher prevalence of the aggressive phenotype \citep{hoffman_low_2000, schatzl_highgrade_2001, damico_lower_2002, massengill_pretreatment_2003, schnoeller_circulating_2013}. It has been therefore speculated that low testosterone levels may promote the development of aggressive prostate cancer \citep{hsing_obesity_2007, freedland_obesity_2007}. Lastly, in the PCPT it was observed that finasteride, a drug that lowers dihydrotestosterone levels, decreased overall prostate cancer risk, but at the same time it increased the risk of high-grade tumors (Gleason score 7--10) \citep{thompson_influence_2003}. In the long-term analyses with 18 years of follow-up, finasteride was again observed to reduce the risk of prostate cancer \citep{thompson_longterm_2013}. This reduction was entirely due to fewer low-grade tumors (Gleason score 2--6), as high-grade prostate cancers were still more common in the finasteride arm. The relative increase of high-grade tumors, however, decreased from 27\% in the primary study to 17\% in the long-term study. 

Third, obesity is a state of chronic inflammation mediated through altered levels of adipokines, such as leptin (a potent inflammatory agent) or adiponectin (an anti-inflammatory adipokine). Leptin, which is elevate in obesity, has been observed to have a pro-tumor potential in vitro \citep{somasundar_prostate_2004}, but epidemiologic studies observed no evidence of an association with prostate cancer risk or tumor stage at prostatectomy  \citep{freedland_serum_2005, baillargeon_obesity_2006, li_25year_2010}. Conversely, adiponectin has anti-tumor properties and its serum levels are decreased in obese individuals \citep{freedland_adiponectin_2010, dalamaga_role_2012}. Adiponectin has been found to be inversely associated with metastatic prostate cancer incidence and mortality \citep{li_25year_2010}. Moreover, adiponectin levels have been observed to be inversely associated with tumor's histologic grade and disease stage among men diagnosed with prostate cancer \citep{goktas_prostate_2005}.


\subsubsection{BMI during early adulthood}

The physiologic changes during the developmental stages of the prostate and immediately thereafter --- when the prostate may be more susceptible to endogenous and exogenous carcinogenic exposures --- may play an important role in tumor initiation and development \citep{hsing_hormones_1996, giovannucci_height_1997, sutcliffe_prostate_2013}. However, the biologic mechanisms that may explain a possible link between prostate cancer risk and obesity during childhood, puberty and early adulthood are unclear.

Adiposity during early life and adolescence, which has been observed to often persist during adulthood \citep{the_association_2010}, has been found to be associated with delayed pubertal development in boys \citep{wang_obesity_2002}. %and to lower pubertal androgen levels \citep{giovannucci_height_1997}. 
Due to the fact that puberty is associated with a steep increase in IGF-1 levels \citep{keenan_androgenstimulated_1993, juul_serum_1994}, a delayed  pubertal development could, in theory, lead to a lower cumulative exposure to IGF-1 and/or a lower exposure during those ages that are crucial for prostate development. In addition, obesity during earlier stages of life is linked with later diabetes incidence, which in turns has been consistently observed to be inversely associated with the risk of prostate cancer \citep{jiangang_diabetes_2015}. 

These speculative biological mechanisms could partially explain the inverse association that we observed between BMI at age 30 years and incidence of advanced prostate cancer and prostate cancer mortality. 

\subsection{Detection bias}


There is a number of factors that could make prostate cancer detection more difficult in obese men. This relationship between body adiposity and detection sensitivity could lead to an apparent `effect modification' by aggressiveness of the disease at diagnosis \citep{garcia-closas_invited_2015}. Therefore, the heterogeneity in the association between BMI during middle-late adulthood and the incidence of prostate cancer by subtype of the disease might have non-causal explanations.

First, obese men have lower mean PSA values \citep{banez_obesityrelated_2007}, which in turns leads to a reduction in the rate of PSA-driven biopsies. The reason why obese men have lower PSA levels on average is still unclear. One explanation is that obese men have lower testosterone levels, leading to less PSA production. It has also been hypothesized that this is due to increased blood volume in obese subjects causing PSA hemodilution, since no evidence of an association between mean PSA mass and BMI was observed in some studies \citep{banez_obesityrelated_2007, grubb_serum_2009}. 

Second, a thorough digital rectal examination is more difficult in obese men and its predictive value in prostate cancer detection has been observed to be modified by obesity \citep{chu_predictive_2011}, which could result in missed cancers.

Third, several studies have suggested that obese men have larger prostates on average \citep{dahle_body_2002, freedland_obesity_2006}, which reduces the likelihood of finding cancer at biopsy \citep{freedland_obesity_2006}. Furthermore, most prostate cancers detected by PSA screening are very small and they cannot be visualized with conventional imaging, making prostate biopsy ``analogous to looking for a needle in a haystack'' \citep{buschemeyer_obesity_2007}. Therefore, prostatic enlargement in obese men would make biopsy detection even more complicated, all other things being equal \citep{kranse_predictors_1999}. 

All these factors combined could potentially allow prostate cancer growth to continue undetected. This would eventually result in  a higher occurrence of advanced disease in obese men as compared with normal-weight men and, at the same time, in a lower occurrence of localized disease.

To what extent detection bias explains the results reported in epidemiologic studies, including ours, it is unknown. However, even in the pre-PSA era obesity  was positively associated with prostate cancer mortality \citep{rodriguez_body_2001}. Furthermore, obesity has been observed to be associated with decreased odds of low-grade prostate cancer and, at the same time, increased odds of high-grade prostate cancer even when all men underwent biopsy \citep{gong_obesity_2006}. Detection bias is therefore unlikely to fully explain the association between BMI and prostate cancer mortality, and the subtype-specific associations with prostate cancer incidence.

Lastly, in our data we observed no evidence of heterogeneity in the associations by county of enrollment (table \ref{table:paper1_heterogeneity}).\footnote{County of enrollment reflected possible different degrees of PSA-screening uptake. See section \ref{section:materials_incpca}.} Although this analysis is obviously sub-optimal --- men enrolled in one county may have in fact moved somewhere else during the follow-up time --- it provides no evidence in support of the hypothesis that the observed associations are due to detection bias. 

\subsection{Body mass index as surrogate of obesity}

The usage of BMI as a proxy for obesity --- ``a condition characterized by the excessive accumulation and storage of fat in the body''\footnote{\href{http://www.merriam-webster.com/dictionary/obesity}{\texttt{http://www.merriam-webster.com/dictionary/obesity}}} --- has some limitations. At the same time, thanks to its simplicity and cost-effectiveness, BMI is likely the most commonly used measure of obesity in large-scale epidemiologic studies \citep{michels_does_1998}.

The major shortcoming of this measure is probably that single individuals' specific composition and body shape do not enter into BMI calculations. For example, very muscular men with little body fat can have a high BMI, which would be wrongly interpreted as an indication of obesity. However, these types of men arguably constitute  a limited proportion of the men enrolled in most studies, with an exception possibly being the Swedish Construction Workers cohort \citep{stocks_blood_2010}, which was included in the meta-analysis of \citetalias{discacciati_body_2012}. 

Although BMI may be inadequate to measure body adiposity for a single subject, it remains a reasonably accurate measure of body adiposity in populations \citep{flegal_comparisons_2009}. However, a systematic tendency to over-report height and under-report weight, resulting in BMI being biased downwards, has been repeatedly observed \citep{connorgorber_comparison_2007}. Perhaps not surprisingly, this tendency to under-report weight has been observed to be more common among overweight and obese individual than among normal-weight individuals \citep{bostrom_socioeconomic_1997, kovalchik_validity_2009}.

The consequences of this measurement error depend on the ``true'' unknown underlying association that is being examined. For example, for advanced and fatal prostate cancer risk --- assuming that the ``true'' association with BMI during middle-late adulthood is positive ---  a systematic under-reporting of BMI in obese men would lead to attenuated associations (RRs closer to the unity). 

%More objective methods for assessing body composition include bioelectrical impedance analysis, dual energy X-ray absorptiometry, and dilution techniques \citep{lee_assessment_2008}, but these are rarely employed in large-scale epidemiologic studies mainly due to logistic and financial reasons.













