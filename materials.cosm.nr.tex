% ---------------------------------------------------------
% Project: PhD KAPPA
% File: materials.cosm.nr.tex
% Author: Andrea Discacciati
%
% Purpose: Section COSM + National Registers (materials)
% ---------------------------------------------------------

\citetalias{discacciati_body_2011} and the additional results for \citetalias{bellavia_using_2015} presented in this thesis are based on the population-based longitudinal COSM study. The meta-analysis in \citetalias{discacciati_body_2012} is based on aggregated data extracted from studies identified through a search of computerized databases, and by reviewing reference lists.

\section{The Cohort of Swedish Men}
\label{section:cosm}
The principal aim of the COSM\footnote{\href{https://clinicaltrials.gov/ct2/show/NCT01127711}{\texttt{https://clinicaltrials.gov/ct2/show/NCT01127711}}} is to investigate the relations between a number of lifestyle and diet factors and the incidence of several chronic diseases, including cancer \citep{harris_swedish_2013}.

The population-based COSM was established in the fall 1997, when all men born between 1918 and 1952 residing in Västmanland and Örebro counties in central Sweden ($n= 100303$) received an invitation to participate in the study, along with a self-administered questionnaire. This questionnaire included questions about height, current body weight, body weight across the life course (at ages 20, 30, 40, 50, 60, and 70), education, physical activity (at ages 15, 30, 50, and current), smoking habits, family history of prostate cancer, and diet.

A total of 48,850 men returned the questionnaire (49\%). After excluding those participants who reported an incorrect/incomplete personal identificaiton number ($n= 205$),\footnote{The number of men who returned the questionnaire reported in \citetalias{discacciati_body_2011} is equal to $48645=48850-205$.} returned an incomplete questionnaire ($n= 92$),  died before 1 January 1998 ($n= 55$), or had a prevalent cancer diagnosis (except non-melanoma skin cancer) ($n= 2592$), the final cohort consisted of 45,906 men. The follow-up period started the 1st of January 1998.

The COSM is comparable to the general Swedish population of men aged 45--79 years in 1997 with regards to age distribution, education level, and proportion of overweight (BMI > 25 \kgmsq) \citep[see table \ref{tab:cosmswedishpop}, reproduced from][]{harris_swedish_2013}. %TODO: Add paper  Kuskowska-Wolk?

\begin{table}[ht]
\centering
\begin{threeparttable}
\caption[Comparison between COSM and Swedish population, 45--79 years of age, 1997]{Comparison of the COSM with the Swedish male population aged 45--79 years in 1997, regarding age distribution, educational level, and proportion of overweight \citep{harris_swedish_2013}.}
\label{tab:cosmswedishpop}
\begin{tabular}{lcc}
\hline
                           & \begin{tabular}[c]{@{}c@{}}{\bf COSM study population} \\ {\bf aged 45--79 years in 1997}\end{tabular} &  \begin{tabular}[c]{@{}c@{}}{\bf Swedish male population} \\ {\bf aged 45--79 years in 1997}\tnote{a}\end{tabular} \\ \hline
\multicolumn{3}{l}{{\bf Age groups (years)}}                                                                                                    \\ \hline
Total (n)                  & 45,906          & 1,594,952                                                                                       \\
45--49 (\%)                & 15.9            & 19.5                                                                                            \\
50--54 (\%)                 & 18.8            & 20.8                                                                                            \\
55--59 (\%)                 & 15.9            & 15.6                                                                                            \\
60--64 (\%)                & 13.1            & 12.5                                                                                            \\
65--69 (\%)                & 14.1            & 11.6                                                                                            \\
70--74 (\%)                & 12.4            & 10.8                                                                                            \\
75--79 (\%)                & 9.9             & 9.2                                                                                             \\ \hline
\multicolumn{3}{l}{{\bf Education, ages 48--74 years}\tnote{b}}                                                                                         \\ \hline
Total (n)                  & 41,382          & 144,585                                                                                         \\
$\le$ 12 years (\%)        & 82.3            & 77.1                                                                                            \\
\textgreater 12 years (\%) & 17.3            & 21.0                                                                                            \\ \hline
\multicolumn{3}{l}{{\bf Overweight (BMI \textgreater 25 \kgmsq), by age groups (years)}}                                                               \\ \hline
45--54 (\%)                & 54.5            & 57.2                                                                                            \\
55--64 (\%)                & 59.1            & 60.3                                                                                            \\
65--74 (\%)                & 59.8            & 57.0                                                                                            \\
75--84 (\%)                & 47.5            & 43.0                                                                                            \\ \hline
\end{tabular}
\begin{tablenotes}
\item [a] \footnotesize Data from SCB.
\item [b] \footnotesize Educational level reported are for those $\le$ 74 years of age since data from SCB were not available for older ages.
\end{tablenotes}
\end{threeparttable}
\end{table}
The protocol of this study has been approved by the relevant ethical committee. All subjects gave full informed consent to participate in this study.


\section{The national registers}

\subsection{The Swedish Cancer Register}

The Swedish Cancer Register (SCR), maintained by the NBHW, was founded in 1958 and covers the whole population in Sweden. Epidemiologic research is one of the specific objectives of this register. The information available from the SCR includes patient data (personal identification number, age, sex, and residence), medical data (tumor site, histological type, and date of diagnosis), and follow-up data (including date and cause of death). Health care providers are obliged by law to report newly diagnosed cancers to the SCR. Completeness of this register was observed to be high \citep{barlow_completeness_2009}.


\subsection{The National Prostate Cancer Register}

From 1998 the NPCR includes all the incident cases of prostate adenocarcinomas diagnosed in Sweden. One of the primary aims of the NPCR is to provide data for clinical research. This register includes detailed information on tumor stage (according to the TNM staging system), Gleason score, and PSA serum level at diagnosis \citep{vanhemelrijck_cohort_2013}. It has been estimated that, during the period 1998--2012, a total of 98\% of men diagnosed with prostate cancer  registered in the SCR had also been registered in the NPCR  \citep{tomic_evaluation_2015}.

\subsection{The Cause of Death Register}

Starting from 1953, the NBHW maintains the Cause of Death Register (CDR). This register contains information on the date and underlying cause of death of all Swedish citizens, irrespective of whether they died in Sweden or abroad. The CDR is complete from 1 January 1969. The reliability of official cause-of-death statistics of prostate cancer patients in Sweden was observed to be reasonably high \citep{fall_reliability_2008, godtman_high_2011}. 

Deaths whose underlying cause was attributed to prostate cancer (code 61 according to the International Classification of Diseases, 10th revision) are equally referred to in this thesis  as `prostate cancer deaths' or `fatal prostate cancer'. 
