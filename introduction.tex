% ---------------------------------------------------------
% Project: PhD KAPPA
% File: introduction.tex
% Author: Andrea Discacciati
%
% Purpose: Introduction
% ---------------------------------------------------------

\chapter{Introduction}

\begin{quote}
\textit{``Epidemiological studies provide the only definitive information on the degree of cancer risk to man. Since malignant diseases are clearly of multifactorial origin, their investigation in man has become increasingly complex, and epidemiological and statistical studies on cancer require a correspondingly complex and rigorous methodology.''}
\par\raggedleft---Lorenzo Tomatis\footnote{Lorenzo Tomatis (*1929--\textdagger2007) was the Director from 1982 until 1993 of the International Agency for Research on Cancer in Lyon, France. This quotation is taken from the foreword he wrote for the book ``Statistical methods in cancer research. Volume 2 — The analysis of cohort studies'' \citep{breslow_statistical_1987}.}
\end{quote}

\bigskip


Prostate cancer was the second most common cancer in men worldwide and the most common one in developed countries in 2012 \citep{ferlay_cancer_2015}, yet its etiology remains poorly understood. To date, the only established risk factors are those that are non-modifiable: age, family history of the disease, and race/ethnicity \citep{gronberg_prostate_2003}. %Therefore, the identification of modifiable risk factors which might prevent prostate cancer development and progression is of paramount importance.

The identification of potential modifiable risk factors is  complicated by the considerable biologic heterogeneity of the disease --- ranging from indolent to potentially lethal tumors --- suggesting different etiologies and distinct entities \citep{discacciati_lifestyle_2014, jahn_high_2015}. %As a consequence, epidemiologic studies should focus on the analysis of prostate cancer separately by its aggressiveness . 

%which is reflected by diverging risk factors patterns between aggressive and indolent prostate cancer \citep{jahn_high_2015}. %Men who develop prostate cancer suffer significant impairments in quality of life, both attributable to the disease itself and to side effects of treatment. 

Obesity is a major global public health concern, with 205 million men worldwide estimated to be obese. This obesity epidemic is particularly severe in developed countries, where, for example, as much as 20\% of men living in Western Europe and 30\% in the U.S. were estimated to be obese \citep{finucane_national_2011}.

Since body fatness is related to hormonal and metabolic changes and given that prostate cancer is a hormone-related cancer, the hypothesis of an association between obesity and prostate cancer risk --- possibly depending on the aggressiveness of the disease --- has been repeatedly formulated \citep{hsing_obesity_2007}. 

Elucidating the possible association between obesity and prostate cancer is not only important to unravel the etiology of the disease, but it is also of public health significance, as these two medical conditions affect large proportions of the male population. In addition, the fact that obesity is a largely preventable condition might provide strategies for reducing prostate cancer incidence and mortality.

\bigskip
\bigskip

%Rather, it is also of paramount public health significance, as these two medical conditions affect large proportions of the male population. 
   
As the words by Lorenzo Tomatis remind us, epidemiologic investigation cannot be separated from epidemiologic methods. Likewise, the two aims of this thesis are intertwined. 

First, this thesis focuses on elucidating the association of body fatness measured during early and middle-late adulthood with localized and advanced prostate cancer incidence and mortality. This is done by analyzing primary data from a large population-based cohort study \citepalias{discacciati_body_2011} and by summarizing the existing epidemiologic evidence in form of aggregated data \citepalias{discacciati_body_2012}.

Second, this thesis deals with some methodological aspects related to the analysis of primary and aggregated data. In particular, \citetalias{bellavia_using_2015} extends the use of quantile regression for censored data to those situations where attained age is the time scale of interest, \citetalias{discacciati_interpretation_2015} clarifies the appropriate use and interpretation of risk and rate advancement periods, while \citetalias{discacciati_goodness_2015} presents relevant methods for assessing the goodness of fit of dose--response meta-analysis models for binary outcomes.
 
%This thesis focuses on elucidating the possible associations of body fatness measured during early and middle-late adulthood with localized and advanced prostate cancer incidence, as well as with prostate cancer mortality. This is done by analyzing primary data coming from a large population-based cohort study of around 45,000 Swedish men and by summarizing the existing epidemiologic evidence in form of aggregated data.

%Lastly, as the words written by Lorenzo Tomatis remind us, epidemiologic investigation cannot be separated from epidemiologic methods. Therefore, this thesis also addresses some methodological topics that are strictly related to the analysis of primary and aggregated data.



