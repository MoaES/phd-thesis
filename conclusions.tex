% ---------------------------------------------------------
% Project: PhD KAPPA
% File: conclusions.tex
% Author: Andrea Discacciati
%
% Purpose: Conclusions
% ---------------------------------------------------------

\chapter{Conclusions}

The results presented in this thesis contribute to the body of scientific evidence regarding the association between BMI during early and middle-late adulthood and prostate cancer incidence and mortality. Furthermore, this thesis contributes to the advancement of the epidemiologic field by extending the use of quantile regression for censored data to those situations where attained age is the time scale of interest, by clarifying the appropriate use and interpretation of RAP, and by proposing useful and relevant methods to assess the goodness of fit of dose--response models in research synthesis.

More specifically we conclude the following:

\begin{itemize}

\item In a large population-based cohort of Swedish men, BMI measured during middle-late adulthood was inversely associated with the incidence of  localized prostate cancer. At the same time, BMI was directly associated with the incidence of advanced prostate cancer and with prostate cancer mortality. BMI during early adulthood was only weakly inversely associated with the incidence of advanced prostate cancer and with prostate cancer mortality \citepalias{discacciati_body_2011}.

\item Similar results regarding the dual dose--response association between BMI during middle-late adulthood and the incidence of localized and advanced prostate cancer were observed by summarizing the published epidemiologic evidence. This supports the hypothesis of etiological heterogeneity of prostate cancer in relation to obesity during middle-late adulthood \citepalias{discacciati_body_2012}.

\item The use of quantile regression for censored data can be extended to those situations where the time scale of interest is attained age at the event instead of follow-up time. In particular, in the presence of delayed entries, Laplace regression can be used to model percentiles of age at the event by conditioning on baseline age \citepalias{bellavia_using_2015}.

\item The misconceptions appeared in the literature radically changed the meaning of RAP. Moreover, we showed how this measure is extremely sensitive to the form of the age-disease dependence in the rate or risk model. As a result, RAP can make more harm than good if misinterpreted or estimated from misspecified models \citepalias{discacciati_interpretation_2015}. 

\item Goodness of fit of dose--response meta-analysis models should be routinely assessed. The tools illustrated in this thesis prove useful to test, quantify, and visually display the fit of dose–response meta-analysis models, while taking into account the correlation structure of the study-specific logRRs \citepalias{discacciati_goodness_2015}.

\end{itemize}